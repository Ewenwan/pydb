% $Id: libpydb.tex,v 1.99 2006/10/30 15:28:15 rockyb Exp $

% A note on labels. First, labels get used has HTML names if they don't
% have colons in them. Second, the last label inside a (sub)section will
% get used as a (sub)section head. Consequently we use colons for
% labels that aren't sections head and don't use colons if they are
% in a section head.

\chapter{The Extended Python Debugger \label{chapter-pydb}}

\declaremodule{standard}{pydb} \modulesynopsis{The Extended Python
debugger for interactive interpreters.}

The extended Python debugger builds on work done in the standard
\ulink{\module{Python
Debugger}}{http://docs.python.org/lib/module-pdb.html} ({\tt
pdb.py}). However unless there is reason not to, we follow the
\ulink{the GNU debugger}{http://www.gnu.org/software/gdb} ({\tt gdb})
command set and semantics rather than {\tt pdb}'s. In some cases where
semantics differ, e.g for {\tt clear} and noted in below in context,
the {\tt gdb} semantics is used.

{\bf On \emph{not} inventing yet another interface}:\\ In extending
the command set and functionality, we've used \ulink{the GNU
debugger}{http://www.gnu.org/software/gdb} ({\tt gdb}) as a guide for
many reasons. Because of gdb's longevity and pervasiveness, the
command set is likely to be complete and it is likely to be familiar.
(I also base my \ulink{GNU Bash
debugger}{http://bashdb.sourceforge.net} {\tt bashdb} on this command
set and to some extent my \ulink{GNU Make
debugger}{http://bashdb.sourceforge.net/remake}). Thus the learning
curve is reduced for people familiar with one of these, and they are
less likely to get confused when switching between debuggers.

But it may make it easier to teach \emph{programs} as well. I digress for a
little history.

When I first thought about adding the bash debugger into the GUI
interface, I looked around and saw the GNU/GPL program
\ulink{\code{ddd}}{http://www.ddd.org} which already supported a
number of debuggers already, such as \code{gdb}, and the debugger for
Perl, and even one for Python! (In fact this debugger's name comes
from that debugger). To instruct \code{ddd} for a new debugger, there
are a myriad of little details \code{ddd} needs to know
about. Basically you need tell \code{ddd} how to issue a ``step''
command, or how to set a breakpoint and how find out if the command it
issued worked. Since I copied \code{gdb}, I basically told \code{ddd}
to handle a number of these constructs (e.g., stepping) like
\code{gdb}, which it already knew about. There were a few places where
I told \code{ddd} not to follow \code{gdb} but Perl instead because
the paradigm was more like a scripting language than a compiled
language. But in the end, adding support for the bash debugger inside
\code{ddd} was much more straightforward and required much less
thought than if I had invented my own debugger command set.

After this was done and I fired up \code{ddd}, I noticed that, when my
cursor was hovering over some of the buttons, short descriptions for
the command were given. Furthermore it created a button called
``customize bash'' which when clicked would pop up a window to set
various debugger parameters! But I hadn't added a box widget for
customization or modified any code for using tool tips. How did
\code{ddd} do this?

Because I had copied the output format of \code{gdb}'s \code{info},
\code{set} and \code{show} commands, \code{ddd} ran these commands on
its own and parsed the output; it then used that output to form tool
tips and create customization boxes.

In responses to a preliminary posting to \code{comp.lang.python}
asking why the Python debugger was different from other debuggers, a
number of people indicated that it didn't matter since they did not use
the standard Python debugger, or did not use it much.  To some extent,
I wonder if this is not a chicken-and-egg problem: is the debugger
lacking in usefulness because people don't use it much or do people
not use the debugger because it is lacking in usefulness?

I'm not sure, but if the standard Python debugger is little used,
keeping compatibility is not important.

So, in cases where the standard Python
debugger was incompatible with \code{gdb}, the \code{gdb} commands
have been used.

{\bf Least action: a design principle for debuggers in general}:\\ By
necessity, debuggers change the operation of the program they are
debugging. And this can lead to unexpected and unwanted
differences. It has happened so often that the term
\ulink{http://en.wikipedia.org/wiki/Heisenbug}{``Heisenbugs'} was
coined to describe the situation where the adding the use of a the
debugger (among other possibilities) changes behavior of the program
so that the bug doesn't manifest itself anymore.

Of course a debugger, and this one in particular, tries hard to make
itself transparent when not asked to do its thing. But there can be
unavoidable differences. One such noticeable difference, mentioned in
conjunction with \code{set_trace()} (see \ref{set-trace}), is the
speed at which the debugger runs. Another difference involves
name-space issues since some of the debugger may live inside the
debugged program.

Consequently, a general principle as a debugger writer (and a
principle used in this debugger) is: \emph{bring in services only when
needed or requested.}  For example, in out-of-process debugging some
sort of communication mechanism is needed, e.g. a socket. However
there are many programs that one might want to debug which might not
use sockets and might not need out-of-process debugging. For those, a
debugger should therefore not add a requirement on sockets.  And even
then, perhaps not \emph{until} the service which uses it (remote
debugging) is requested.

\section{Invoking the Debugger \label{pydb-invocation}}

First it should be noted that in various situations and for various
reasons you might not be able to call the debugger directly from a
command line or at the outset of the program. That's okay. See section
\ref{subsection-calling-pydb-inside-program} for how to invoke the
debugger from inside your running Python program. Also, see section
\ref{subsection-post-mortem} for how to invoke after an error is
encontered and you have a traceback.

Of the many additions to the standard Python debugger, pdb, three will
be mentioned here. 

First, this debugger should install itself somewhere in your command
path, usually as \code{pydb} so you don't have to invoke it as an
argument to the \code{python} command.\footnote{However there
\emph{is} one special case where you may want to invoke via Python. If
you want warnings to appear as errors, one way to do this is
to run the \code{pydb.py} program passing the \code{-Werror} option to
Python.}

Second the extended debugger supports command switches courtesy of
\ulink{\module{optparse}}{http://docs.python.org/lib/module-optparse.html}.
In particular \code{pydb} has the two very common options
\samp{--help} to show what options are available and \samp{--version}
to report the version that is installed. 

Third, you need not supply a script name to debug at the
outset. Usually though you will want to give the name of a script to
debug and after that you may want to pass options for this script.

Thus the general form of invoking the debugger is:

\code{pydb} \optional{\var{debugger-options}\ldots}
\optional{\var{python-script} \optional{\var{script-options}\ldots}}

\var{python-script} should be the Python source (usually has extension
\code{.py} if any), and not a compiled or optimized Python program.

In contrast to running a program from a shell (or using the \code{gdb}
debugger), no path searching is performed on
\var{python-script}. Therefore \var{python-script} should be explicit
enough (include relative or absolute file paths) so that the debugger
can read it as a file name. 

Similarly, the location of the Python interpreter used for the script
will not necessarily be the one specified in the magic field (the
first line of the file), but will be the Python interpreter that the
debugger specifies. (In most cases they'll be the same and/or it won't
matter.)

If you are familiar with the stock debugger \code{pdb}, you may feel
the urge to write a wrapper script. As indicated above, you shouldn't
have to if \code{pydb} got installed properly and a symbolic link was
inserted.

That said, if installation didn't work or you want to create a shell
wrapper script, sure, you can. Here's an example from a SUSE GNU/Linux
box.

Looking at the output from \code{make install} I see pydb was
installed in \code{/usr/lib/python2.4/site-packages}. Therefore save
the following in a script and put it somewhere in your \code{PATH}
(and make it executable):

\begin{verbatim}
  #!/bin/sh
  export PYTHONPATH="/usr/lib/python2.4/site-packages/:$PYTHONPATH:"
  /usr/lib/python2.4/site-packages/pydb/pydb.py $*
\end{verbatim}

A detailed list of options is given next. 

\subsection{Debugger Command Options ({\tt --trace}, {\tt
    --output}, {\tt --command}, {\tt --nx}, \ldots)\label{subsection-switches}}

Many options have both a short and a long version. For example,
\samp{-x} is the short version while \samp{--command} is the long
version.

\begin{description} 

\item[--{}--basename]

Report file locations as only the base filename, and omit the
directory name. This is useful in running regression tests.

\item[--{}--batch]

Normally the debugger is entered before the debugged script is
executed. The user sets breakpoints or starts interactively stepping
through the program. However, if you want to start the script running
without any interactive behavior from the debugger, use this option
for example, if you know that a script will terminate with an
exception which causes the debugger to be entered at that point. The
\code{--trace} option implicitly sets this option.

\item[--{}--cd \var{directory}]

Run \code{pydb} using \var{directory} as its working directory,
instead of the current directory.

\item[--{}--command=\var{filename}\code{\Large{|}}--x \var{filename}]\label{switch:command}

Run debugger script \var{filename}. This script is run after
the user's \code{.pydbrc} file.

\item[--{}--exec=\var{command-list}\code{\Large{|}}--e \var{command-list}]\label{exec:command}

Run debugger commands \var{command-list}. Commands should be separated
by ``\code{;;}''---the same as you would do inside the debugger. You may
need to quote this option to prevent command shell
interpretation, e.g. \code{--exec "break 20;; continue"}.

\item[--{}--nx\code{\Large{|}}]\label{switch:nx}

Before execution is started, a debugger configuration file
\code{.pydbrc} is run. In some situations, for example regression
testing the debugger, you want to make sure that such configuration
files are not run and this option will do that.

\item[--{}--output=\var{filename}]

Write the normal output (\samp{stdout}) to the file
\var{filename}. Useful when running a Python script without access to
a terminal.

\item[--{}--error=\var{filename}]

Write the error output
(\samp{stderr}) to file \var{filename}. Useful in
 running a Python script without access to a terminal.

\item[--{}--threading]\label{switch:threading}
Allow thread debugging. See \ref{subsection-thread}.

\item[--{}--trace]\label{switch:trace}

POSIX-style line tracing is available. In POSIX shells the short
option for this is \code{-x}; however since we follow \code{gdb}
conventions \code{-x} is used as a short option for
\code{--command}. When line tracing is turned on, each location (file
name and linenumber) is printed before the command is executed. This
option can be used in conjunction with the \code{--output} and
\code{--error} options described above when a terminal is not
available or when not running interactively. The corresponding
debugger command is \samp{set linetrace on}. See
\ref{command:linetrace} for more information.

\end{description}

\subsection{Startup files ({\tt .pydbrc})\label{subsection-startup-files}}

If a file \code{.pydbrc}
\indexii{.pydbrc}{file}\indexiii{debugger}{configuration}{file} exists
in the user's home directory or in the current directory, it is read
in and executed as if it had been typed at the debugger prompt.  This
is particularly useful for aliases.  If two files exist, the one in
the home directory is read first and aliases defined there can be
overridden by the local file. Finally you can specify a command file
to be read when invoking \code{pydb} and this is run last. See
\ref{switch:command} for information on how to run a command file.

Sometimes you may not want to run startup files. For example,
you may have a special installation script that uses the debugger and
want to make sure a user's profile doesn't get in the way. See
the \samp{--nx} command option, \ref{switch:nx}.

For tracking down problems with command files, see the \samp{set
cmdtrace on} debugger command, \ref{command:cmdtrace}. To run a
debugger command script inside the debugger see the \samp{source}
command, \ref{command:source}.

\section{Debugger Commands\label{pydb-commands}}

In this section we describe debugger commands which can be used when
the debugger is run as a standalone program.

Most commands can be abbreviated to one or two letters;
e.g., \samp{h(elp)} means that either \samp{h} or \samp{help} can be
used to enter the help command (but not \samp{he}, \samp{hel},
\samp{H}, \samp{Help}, or \samp{HELP}).  Arguments to commands must
be separated by whitespace (spaces or tabs).  Optional arguments are
enclosed in square brackets (\samp{[]}) in the command syntax; the
square brackets must not be typed.  Alternatives in the command syntax
are separated by a vertical bar (\samp{|}).

Entering a blank line repeats the last command entered.  Exception: if
the last command was a \samp{list} command, the next 11 lines are
listed.

Commands that the debugger doesn't recognize are assumed to be Python
statements and are executed in the context of the program being
debugged.  Python statements can also be prefixed with an exclamation
point (\samp{!}).  This may be a good way to inspect the program being
debugged; it is even possible to change a variable or call a function.
When an exception occurs in such a statement, the exception name is
printed but the debugger's state is not changed.

The debugger supports aliases.  Aliases can have parameters which
allow a certain level of adaptability to the context under
examination. See \ref{command:aliases}.

{\bf Debugger Prompt}:\label{debugger:prompt}\\
By default the debugger's prompt string is \samp{(Pydb) } with 
a trailing blank. Recursive invocations using the
\samp{debug} command strip off the trailing blanks, add a layer of
parenthesis around the string, and add a trailing blank. For example,
for the default prompt the first debug invocation will be
\samp{((Pydb)) }.

\emph{There's currently a bug in the code where specified trailing
blanks are chopped. Furthermore the prompt may change in the future to
add a history number. It is generally not advisable to change the
prompt.}

If you do need to change the prompt see \ref{command:prompt}.

Multiple commands may be entered on a single line, separated by
\samp{;;}.  (A single \samp{;} is not used because it is
the separator for multiple commands in a line that is passed to
the Python parser.)
No intelligence is applied to separating the commands;
the input is split at the first \samp{;;} pair, even if it is in
the middle of a quoted string.

\subsection{Status and Debugger Settings ({\tt info}, {\tt set}, {\tt show})\label{subsection-status}}

An {\tt info} command shows things about the program being debugged. A
{\tt set} command modifies parts of the debugger environment. You can
see these environment settings with the {\tt show} command.

The subobptions to \code{info}, or \code{set} (or \code{show}) don't
can be abbreviated to any prefix that uniquely specifies them. For
example \code{info li} is a valid abbrevation for \code{info line}
while \code{info l} is not since there is another subcommand
(\code{locals}) which also starts with an `l'.

In all of the set options that take ``on'' or ``off'' parameters, you
can also use 1 for ``on'' and 0 for ``off.''

Each command has a corresponding {\tt show} command to show the current
value. See \ref{subsubsection-show} for these counterparts.

If a
\ulink{\module{readline}}{http://docs.python.org/lib/module-readline.html}
module is available, \code{pydb} can keep track of the commands you
type during your debugging sessions, so that you can be certain of
precisely what happened.  The \code{set history} commands to manage
the command history facility.

POSIX-style line tracing is available and the \code{set linetrace}
commands can be used to control that.

You may want to save the output of \code{pydb} commands to a file.
See the \code{set logging} commands to control \code{pydb}'s logging.

\subsubsection{Info ({\tt info})\label{subsubsection-info}}

Running this command without parameters will print the list of
available info commands. Below is a description of the individual
commands.

\begin{description}

\item[info args]

Show function/method parameters. See \ref{command:info-args}.

\item[info breakpoints]

Show the status of user-settable breakpoints.  Without argument, list
info about all breakpoints.  With an integer argument, list info on
that breakpoint.

The short command for this is \code{L}.

\item[info globals]

Show the global variables. See \ref{command:info-globals}. 

\item[info handle]

Show the signal handling status. See \ref{command:info-handle}. 

\item[info line]

Show the current line number in source file. If a function name is
given, the starting line of the function is reported.

\item[info locals]

Show the local variables. See \ref{command:info-locals}. 

\item[info program]

Show the execution status of the program. The possible status is that
the program is not running (e.g. in post-mortem dump), or the program is
``stoppped'' and if stopped at a breakpoint that is shown as well.

\item[info signal]

Alias for \code{info handle}. See \ref{command:info-handle}. 

\item[info source]

Information about the current Python file.

\end{description}

\subsubsection{Set ({\tt set})\label{subsubsection-set}}

As with subobptions to \code{info}, or \code{set}, \code{show}
subcommands can be abbreviated to any prefix that uniquely
specifies them. For example \code{set lis 5} is a valid abbrevation for
\code{info listsize 5} while \code{set li} is not since there is another
subcommand (\code{linetrace}) which also starts with `li'.

\begin{description}

\item[set basename on\code{\Large{|}}off]\label{command:basename}

When showing filenames print only the basename. This option is useful
in regression testing where the base file names are the same
on different installations even though the directory path may
be different. You may want to use this in other
situtations as well, like showing a debugger session in a manual
such as this one. 

\item[set cmdtrace on\code{\Large{|}}off]\label{command:cmdtrace}
 
Show lines as they are read from the debugger command file (or
\samp{source} debugger command). This is useful in running
regression tests, but it may be helpful in tracking down a problem in
your \code{.pydbrc} file.

\item[set debug-pydb on\code{\Large{|}}off]\label{command:debug-pydb}

Set whether we allow tracing the debugger. This is used for debugging
pydb and getting access to some of its object variables. 

When \code{debug-pydb} is ``on'', the most recent side of the stack
frame will be located somewhere in the debugger; you'll need to adjust
the frame ``up'' to get where the program was before entering the
debugger. (In some versions this and some situations \code{frame 6}
gets you out of the debugger and into the debugged program.

\item[set history filename \var{filename}]\label{command:hist-filename}

Set the filename in which to record the command history.
(the list of previous commands of which a record is kept). The default
file is \verb|~/.pydbhist|.

\item[set history save]

Set saving of the history record on exit.  Use ``on'' to enable the
saving, and ``off'' to disable it.  Without an argument, saving is
enabled.

\item[set history size]

Set the size of the command history, ie. the number of previous
commands to keep a record of. The default is 256.

\item[set linetrace on\code{\Large{|}}off]\label{command:linetrace}

If this is set on, the position (file and linenumber) is shown before
executing a statement. By default this is off. Using the command-line
option \samp{--trace} when invoking \code{pydb} implicitly sets this
on.  For information on \samp{--trace}, see \ref{switch:trace}.

\begin{verbatim}
# pydb --basename --trace hanoi.py 2
(hanoi.py:2): 
+ """Towers of Hanoi"""
(hanoi.py:3): 
+ import sys
(hanoi.py:5): 
+ def hanoi(n,a,b,c):
(hanoi.py:12): 
+ if __name__=='__main__':
(hanoi.py:13): 
+     i_args=len(sys.argv)
(hanoi.py:14): 
+     if i_args != 1 and i_args != 2:
(hanoi.py:18): 
+     n=3
(hanoi.py:20): 
+     if i_args > 1:
(hanoi.py:21): 
+       try: 
(hanoi.py:22): 
+         n = int(sys.argv[1])
(hanoi.py:27): 
+     if n < 1 or n > 100: 
(hanoi.py:31): 
+     hanoi(n, "a", "b", "c")
--Call--
(hanoi.py:5):  hanoi
+ def hanoi(n,a,b,c):
(hanoi.py:6):  hanoi
+     if n-1 > 0:
(hanoi.py:7):  hanoi
+        hanoi(n-1, a, c, b) 
--Call--
(hanoi.py:5):  hanoi
+ def hanoi(n,a,b,c):
(hanoi.py:6):  hanoi
+     if n-1 > 0:
(hanoi.py:8):  hanoi
+     print "Move disk %s to %s" % (a, b)
Move disk a to c
(hanoi.py:9):  hanoi
+     if n-1 > 0:
--Return--
(hanoi.py:9):  hanoi
+     if n-1 > 0:
(hanoi.py:8):  hanoi
+     print "Move disk %s to %s" % (a, b)
Move disk a to b
(hanoi.py:9):  hanoi
+     if n-1 > 0:
(hanoi.py:10):  hanoi
+        hanoi(n-1, c, b, a) 
--Call--
(hanoi.py:5):  hanoi
+ def hanoi(n,a,b,c):
(hanoi.py:6):  hanoi
+     if n-1 > 0:
(hanoi.py:8):  hanoi
+     print "Move disk %s to %s" % (a, b)
Move disk c to b
(hanoi.py:9):  hanoi
+     if n-1 > 0:
--Return--
(hanoi.py:9):  hanoi
+     if n-1 > 0:
--Return--
(hanoi.py:10):  hanoi
+        hanoi(n-1, c, b, a) 
--Return--
(hanoi.py:31): 
+     hanoi(n, "a", "b", "c")
--Return--
(<string>:1): 
+  (bdb.py:366):  run
+                 exec cmd in globals, locals
\end{verbatim}

Adding linetracing output will slow down your
program. Unless single stepping through a program, normally the
debugger is called only at breakpoints or at the call and return of a
function or method. However when line tracing is turned on, the
debugger is called on execution of every statement. 

That said, execution may still be pretty fast. If you want to slow
down execution further, see the following option.

\item[set linetrace delay \var{time}]\label{command:linetrace-delay}

One of the useful things you can do with this debugger if you run it
via a front-end GUI is watch your program as it executes. To do this,
use \samp{set linetrace on} which prints the location before each
Python statement is run. Many front-end GUIs like the one in GNU Emacs
and \code{ddd} will read the location and update the display
accordingly.

There is however one catch---Python runs too fast. So by using this
option you can set a delay after each statement is run in order for
GNU and your eyes to catch up with Python. Specify a floating
point indicating the number of seconds to wait. For example: 

\begin{verbatim}
set linetrace delay 0.5 # 1/2 a second
\end{verbatim}

In my experience half a second is about right.

\item[set listsize \var{lines}]\label{command:listsize}

Sets how many lines are shown by the \code{list} command. See
\ref{command:list}.

\item[set logging]\label{command:logging}

Prints \code{set logging} usage.

\item[set logging on\code{\Large{|}}off]\label{command:logging}

Enable or disable logging.

\item[set logging file \var{filename}]\label{command:log-redirect}

By default, \code{pydb} output will go to both the terminal and the
logfile.  Set \code{redirect} if you want output to go only to the log
file.

\item[set logging overwrite on\code{\Large{|}}off]\label{command:log-overwrite}

By default, \code{pydb} will append to the logfile.  Set
\code{overwrite} if you want \code{set logging on} to overwrite the
logfile instead.

\item[set logging redirect on\code{\Large{|}}off]\label{command:log-redirect}

By default, \code{pydb} output will go to both the terminal and the
logfile.  Set \code{redirect} if you want output to go only to the log
file.

\item[set prompt \var{prompt-string}]\label{command:prompt}

Set debugger's prompt string. By default it is \samp{(Pydb) } with
a trailing space. For information on how the prompt
changes, see \ref{debugger:prompt}.

\emph{There's currently a bug in the code where specified trailing
blanks specified. Furthermore the prompt may change in the future to
add a history number. It is generally not advisable to change the
prompt.}

\item[set sigcheck on\code{\Large{|}}off]\label{command:sigwatch}

Turning this on causes the debugger to check after every statement
whether a signal handler has changed from one of those that is to be
handled by the debugger. Because this may add a bit of overhead to the
running of the debugged program, by default it is set off. However if
you want to ensure that the debugger takes control when a particular
signal is encountered you should set this on.

%\item[set target-address \var{target_addr}]\label{command:target}
%
%This sets up an address so that remote processes can connect this
%process and debug it. See also \code{attach}, \code{target} and
%\code{show target-address}.

\item[set warnoptions \optional{\var{warnoption \ldots}}]

Set the Python warning options that are in effect when a program is started
or restarted. On the command line, these are the \code{-W} options,
e.g. \code{-Werror}, or \code{-We::Deprecation}. However the list
should not contain the leading \code{-W} and options should be
separated with white space only, e.g. don't use commas. 

Examples:
\begin{verbatim}
(Pydb) set warn error e::Deprecation
Warning options used in running a Python program:
	 -Werror, -We::Deprecation
(Pydb) set warnoptions
No warning options have been set.
(Pydb) 
\end{verbatim}

\end{description}

\subsubsection{Show ({\tt show}) \label{subsubsection-show}}

All of the ``show'' commands report some sort of status and all have a
corresponding ``set'' command to change the value. See
\ref{subsubsection-set} for the ``set'' counterparts.

\begin{description}

\item[show args]

Show the argument list that was given the program being debugged or it is
restarted

\item[show basename]

Show short or long filenames

\item[show cmdtrace]

Show the debugger commands before running

\item[show commands]

Show the history of commands you typed. You can supply a command
number to start with, or a `+' to start after the previous command
number shown.  A negative number starts from the end.

This command is available only if a
\ulink{\module{readline}}{http://docs.python.org/lib/module-readline.html}
module is available and supports the history saving.

\item[show debug-pydb]

Show whether the debugging the debugger is set.
See also \ref{command:debug-pydb}

\item[show history]

Generic command for showing command history parameters. The command
history filename, saving of history on exit and size of history file
are shown.

\item[show linetrace]

Show the line tracing status.

\item[show linetrace delay]

Show the delay after tracing each line.

\item[show listsize]

Show the number of source lines \code{pydb} will list by default.

\item[show logging]

Show summary information of logging variables which can be set via
\code{set logging}.

\item[show logging file]

Show the current logging file.

\item[show logging overwrite]

Show whether logging overwrites or appends to the log file.

\item[show prompt]

Show the current debugger prompt.

\item[show sigcheck]

Show whether the debugger checks for reassignment of signal handlers.
See also \ref{command:sigwatch} and \ref{subsection-signal}.

%\item[show target-address]
%
%If you are debugging a remote program, this command can be used to
%show the connection parameters. This command doesn't make sense if
%you are not debugging a remote program. See also \code{set target-address}
%and \code{attach}.


\item[show version]

\item[show warnoptions]

Show the Warning options that are in effect when a program is started
or restarted. On the command-line these are the \code{-W} options,
e.g. \code{-Werror}, or \code{-We::Deprecation}.

\end{description}

\subsection{Breakpoints ({\tt break}, {\tt tbreak}, {\tt clear},
  {\tt commands}, {\tt delete}, {\tt disable}, {\tt condition}, {\tt
  ignore})\label{subsubsection-brkpts}}

\index{breakpoints} A breakpoint makes your program stop at that
point. You can set breakpoints with the \code{break} command and its
variants. You can specify the place where your program should stop by
file and line number or by function name.

The debugger assigns a number to each breakpoint when you create it;
these numbers are successive integers starting with 1.  In many of the
commands for controlling various features of breakpoints you use this
number.  Each breakpoint may be enabled or disabled; if disabled, it
has no effect on your program until you enable it again.

The debugger allows you to set any number of breakpoints at the same
place in your program.  There is nothing unusual about this because
different breakpoints can have different conditions associated with them.

The simplest sort of breakpoint breaks every time your program reaches
a specified place.  You can also specify a \emph{condition} for a
breakpoint.  A condition is just a Boolean expression in your
programming language.  A breakpoint with a condition evaluates the
expression each time your program reaches it, and your program stops
only if the condition is \emph{true}.

This is the converse of using assertions for program validation; in
that situation, you want to stop when the assertion is violated--that
is, when the condition is \emph{false}.

Break conditions can have side effects, and may even call functions in
your program.  This can be useful, for example, to activate functions
that log program progress, or to use your own print functions to
format special data structures. The effects are completely predictable
unless there is another enabled breakpoint at the same address.  (In
that case, \code{pydb} might see the other breakpoint first and stop your
program without checking the condition of this one.)  Note that
breakpoint commands are usually more convenient and flexible than break
conditions for the
purpose of performing side effects when a breakpoint is reached.

Break conditions can be specified when a breakpoint is set, by adding
a comma in the arguments to the \code{break} command. They can also be
changed at any time with the \code{condition} command.

\begin{description}
\item[b(reak) \optional{\optional{\var{filename}:}\var{lineno}\code{\Large{|}}\var{function}\optional{, \var{condition}}}]\label{command:break}

With a \var{lineno} argument, set a break at that line number in the
current file.  With a \var{function} argument, set a break at the
first executable statement within that function.  The line number may
be prefixed with a filename and a colon to specify a breakpoint in
another file (probably one that hasn't been loaded yet).  The file is
searched on \code{sys.path}.  Note that each breakpoint is assigned a
number to which all the other breakpoint commands refer.

If a second argument is present, it is an expression which must
evaluate to true before the breakpoint is honored.

Without an argument, list all breaks, including for each breakpoint:
the number of times that breakpoint has been hit, the current ignore
count, and the associated condition if any. Note that this is a
different behavior from \code{pdb}.

If threading is enabled, you can also specify a thread name. See \ref{command:thread-extensions}.

\item[tbreak \optional{\optional{\var{filename}:}\var{lineno}\code{\Large{|}}\var{function}\optional{, \var{condition}}}]\label{command:tbreak}

Temporary breakpoint, which is removed automatically when it is
first hit.  The arguments are the same as those for \code{break}.

If threading is enabled, you can also specify a thread name. See \ref{command:thread-extensions}.

\item[cl(ear) \optional{\optional{\var{filename}:}\var{lineno}\code{\Large{|}}\var{function}}]

Clear breakpoint at specified line or function.  Argument may be line
number, function name, or `*' and an address.  If a line number is
specified, all breakpoints in that line are cleared.  If a function is
specified, the breakpoints at the beginning of the function are cleared.  If an
address is specified, breakpoints at that address are cleared.

With no argument, clears all breakpoints in the line where the selected
frame is executing.

See also the \code{delete} command below which clears breakpoints by
number. Note that \code{delete} handles some cases that were
previously handled by \code{pdb}'s \code{clear} command.

\item[commands \optional{\optional{\var{bpnumber}}}]

Set commands to be executed when a breakpoint is hit.
Give breakpoint number as the argument after "commands".
With no bpnumber argument, commands refers to the last one set.
The commands themselves follow starting on the next line.
Type a line containing "end" to terminate the commands.

To remove all commands from a breakpoint, type commands and
follow it immediately with \code{end}; that is, give no commands.

Specifying any command resuming execution (currently \code{continue},
\code{step}, \code{next}, \code{return}, \code{jump}, and \code{quit})
terminates the command list as if that command was immediately
followed by \code{end}.  This is because any time you resume execution
(even with a simple next or step), you may encounter another
breakpoint---which could have its own command list, leading to
ambiguities about which list to execute.

If you use the \code{silent} command in the command list, the
usual message about stopping at a breakpoint is not printed.
This may be desirable for breakpoints that are to print a
specific message and then continue.  If none of the other
commands print anything, you see no sign that the breakpoint
was reached.

\item[delete \optional{\var{bpnumber} \optional{\var{bpnumber \ldots}}}]

With a space-separated list of breakpoint numbers, clear those
breakpoints.  Without argument, clear all breaks (but first
ask confirmation).

\item[disable \optional{\var{bpnumber} \optional{\var{bpnumber \ldots}}}]

Disable the breakpoints given as a space-separated list of
breakpoint numbers.  Disabling a breakpoint means it cannot cause
the program to stop execution, but unlike clearing a breakpoint, it
remains in the list of breakpoints and can be (re-)enabled.

\item[enable \optional{\var{bpnumber} \optional{\var{bpnumber \ldots}}}]

Enable the breakpoints specified.

\item[ignore \var{bpnumber} \optional{\var{count}}]

Set the ignore count for the given breakpoint number.  If count is
omitted, the ignore count is set to 0.  A breakpoint becomes active
when the ignore count is zero.  When non-zero, the count is
decremented each time the breakpoint is reached, the breakpoint is not
disabled, and any associated condition evaluates to true.

\item[condition \var{bpnumber} \optional{\var{condition}}]

Condition is an expression which must evaluate to true before
the breakpoint is honored.  If condition is absent, any existing
condition is removed; i.e., the breakpoint is made unconditional.

\end{description}

\subsection{Resuming Execution ({\tt step}, {\tt next}, {\tt
    finish}, {\tt return}, {\tt continue}, {\tt jump})\label{subsubsection-resume}}

``Continuing'' means resuming program execution until the program
completes normally.  In contrast, ``stepping'' means executing just
one statement of the program. When continuing or stepping, the program may stop even sooner, due to a breakpoint or an
exception.

\begin{description}

\item[s(tep) \optional{\var{count}}]\label{command:step}

Execute the current line, stop at the first possible occasion
(either in a function that is called or on the next line in the
current function).

Note that if thread debugging is enabled, the next statement may be in a
different thread.

\item[n(ext) \optional{\var{count}}]\label{command:next}

Continue execution until the next line in the current function
is reached or the function returns.  The difference between \samp{next} and
\samp{step} is that \samp{step} stops inside a called function, while
\samp{next} executes called functions at (nearly) full speed, 
stopping only at the next line in the current function.

Note that if thread debugging is enabled, the next statement may be in
a different thread. There currently is a bug only in thread debugging
where \code{next} can act like \code{step}.

\item[finish]\label{command:finish}

Continue execution until the current function returns.
that point the \samp{retval} command can be used to show the
return value. The short command name is \code{rv}.

See also \ref{command:retval} and \ref{command:return}.

\item[return]\label{command:return}

Make selected stack frame return to its caller. Control remains in the
debugger, but when you continue execution will resume at the return
statement found inside the subroutine or method.  At present we are
only able to perform this if we are in a subroutine that has a
\code{return} statement in it.  See also \ref{command:retval} and
\ref{command:finish}

\item[c(ontinue) \optional{\optional{\var{filename}:}\var{lineno}\code{\Large{|}}\var{function}}]

Continue execution; only stop when a breakpoint is encountered. If a
line position is given, continue until that line is reached. This is
exactly the same thing as setting a temporary breakpoint at that
position before running an (unconditional) \code{continue}.

\item[jump \var{lineno}]

Set the next line that will be executed. available only in the
bottom-most frame.  This lets you jump back and execute code
again, or jump forward to skip code that you don't want to run.

Not all jumps are allowed---for instance it
is not possible to jump into the middle of a \keyword{for} loop or out
of a \keyword{finally} clause.

One common use for the \code{jump} statement is to get out of a
loop. Sometimes the bounds of loops are computed in advance so you
can't leave a loop early by say setting the value of the loop variable

Here's an example demonstrating this:

\begin{verbatim}
pydb ptest.py
(ptest.py:2): 
(Pydb) list
  1  	#!/bin/python
  2  ->	for i in range(1,10):
  3  	    print i
  4  	print "tired of this"
[EOF]
(Pydb) step
(ptest.py:3): 
(Pydb) i=1000
(Pydb) step
1000
(ptest.py:2): 
(Pydb) jump 4
(ptest.py:4): 
(Pydb) step
tired of this
--Return--
--Return--
The program finished and will be restarted
(ptest.py:2): 
(Pydb) 
\end{verbatim}

Not that the assignment of 1,000 to \code{i} took effect, although it
had no effect on termintating the \code{for} loop; \code{jump} was
needed to get out of the loop early.

\end{description}

\subsection{Examining Call Frames ({\tt info args}, {\tt info
    locals}, {\tt down}, {\tt frame}, {\tt up})\label{subsection-frames}}

Each line in the backtrace shows the frame number and the function
name, if it exists and the place in a file where the statement is
located.

Here is an example of a backtrace from a sample Towers of Hanoi
program that is used in regression testing:

\begin{verbatim}
## 0 hanoi() called from file '/tmp/pydb/test/hanoi.py' at line 5
-> 1 hanoi() called from file '/tmp/pydb/test/hanoi.py' at line 6
## 2 in file '/tmp/pydb/test/hanoi.py' at line 29
## 3 in file '<string>' at line 1
## 4 run() called from file '/usr/lib/python2.4/bdb.py' at line 366
\end{verbatim}

The {\tt->} arrow indicates the focus. In the example, I issued an
\samp{up} command which is why the focus is on 1 rather than 0 as it
would normally be after a stop.

There are two ``hanoi'' frames listed because this is a hanoi called
itself recursively. In frame 2 and 3 we don't have a function name
listed. That's because there is none. Furthermore in frame 3 there is
a funny ``in file {\tt '<string>'} at line 1.'' That's because there
isn't even a file assocated with the command. The command issued:

\begin{verbatim}
   exec cmd in globals, locals
\end{verbatim}

This statement can be seen in frame 4. This is a bug which I hope to
fix with a more informative message.

Finally, note that frames 2 and 3 really are not part of the
program to be debugged but are part of the internal workings of the
debugger. It's possible to hide this, but in the open spirit of Python
for now it hasn't been hidden.

\begin{description}

\item[info args]\label{command:info-args}

Show the method or function parameters and their values. 

Here is an example of the output for the backtrace of the hanoi
program shown at the beginning of this section:

\begin{verbatim}
(Pydb) info args
n= 3
a= a
b= b
c= c
(Pydb) 
\end{verbatim}

\item[info locals]\label{command:info-locals}

Show all local variables for the given stack frame. This will include
the variables that would be shown by \samp{info args}.

\item[where\code{\Large{|}}T\code{\Large{|}}bt \optional{\var{count}}]

Print a backtrace, with the most recent frame at the top.  An
arrow indicates the current frame, which determines the context of
most commands.

With a positive number \var{count}, print at most many entries.

An example of a backtrace is given at the beginning of this section.

\item[retval\code{\Large{|}}rv]\label{command:retval}

Show the value that will be returned by the current function. This
command is meaningful only just before a return (such as you'd get
using the \code{finish} or \code{return} commands) or stepping after a
return statement.

To change the value, make an assignment to the variable
\code{__return__}.

See also \ref{command:finish}.

\item[down \optional{\var{count}}]

Move the current frame one level down in the stack trace
(to a newer frame). With a count, which can be positive
or negative, move that many positions.

Note this is the opposite of how \code{pdb}'s \code{down} command works.

\item[up \optional{\var{count}}]

Move the current frame one level up in the stack trace (to an older
frame). With a count, which can be positive or negative,
move that many positions.

Note this is the opposite of how \code{pdb}'s \code{up} command works.

\item[frame \optional{\var{position}}]
Move the current frame to the specified frame number. A negative
number indicates position from the end, so \code{frame -1} moves to 
the newest frame, and \code{frame 0} moves to the oldest frame.

If threading is enabled, you can also specify a thread name. See
\ref{command:thread-extensions}.

\end{description}

\subsection{Examining Data ({\tt print}, {\tt pprint}, {\tt examine}, {\tt info globals})\label{subsection-data}}

\begin{description}

\item[display \optional{\var{format}} \var{expression}]

Print value of expression \var{expression} each time the program
stops.  \var{format} may be used before \var{expression} as in the
"print" command.  \var{format} "i" or "s" or including a size-letter
is allowed, and then \var{expression} is used to get the address to
examine.

With no argument, display all currently requested auto-display
expressions.  Use "undisplay" to cancel display requests previously
made.

\item[undisplay \optional{\var{format}} \var{expression}]

Evaluate the \var{expression} in the current context and print its
value.  \note{\samp{print} can also be used, but is not a debugger
command---it executes the Python \keyword{print} statement.}


\item[p \var{expression}]

Evaluate the \var{expression} in the current context and print its
value. One can also often have an expression printed by just typing
the expression. If the first token doesn't conflict with a debugger
built-in command Python will, by default, print the result same as if
you did this inside a Python interpreter shell. To make things even
more confused, a special case of running an arbitrary Python command
is the \samp{print} command. But note that the debugger command is
just \samp{p}.

So what's the difference? The debugger's print command encloses
everything in a \samp{repr()}, to ensure the resulting output is not
too long.  \emph{\note{Should add info as to how to customize what
``too long'' means}}. So if you want abbreviated output, or are not
sure if the expression may have an arbitrarily long (or infinite)
representation, then use \samp{p}. If you want the output as Python
would print it, just give the expression or possibly use python's
\samp{print} command.

\item[pp \var{expression}]

Like the \samp{p} command, except the value of the expression is
pretty-printed using the 
\ulink{\module{pprint}}{http://docs.python.org/lib/module-pprint.html} module.

\item[examine \var{expression}]

Print the type of the expression and pretty-print its value.  For
functions, methods, classes, and modules print out the documentation
string if any.  For functions also show the argument list.

The examine debugger command in Perl is the model here, however much
more work is needed. Note that \samp{x} is a short name for
``expression'' as it is in Perl's debugger.

\item[info globals]\label{command:info-globals}

Show all global variables. These variables are not just the variables
that a programs sees via a \code{global} statement, but all of them
that can be accessible.

\end{description}

\subsection{Running Arbitrary Python Commands ({\tt debug}, {\tt !})\label{subsection-commands}}

\begin{description}

\item[\optional{!}\var{statement}]

Execute the (one-line) \var{statement} in the context of
the current stack frame.
The exclamation point can be omitted unless the first word
of the statement resembles a debugger command.
To set a global variable, you can prefix the assignment
command with a \samp{global} command on the same line, e.g.:

\begin{verbatim}
(Pydb) global list_options; list_options = ['-l']
(Pydb)
\end{verbatim}

\item[debug \var{statement}]

Enter a recursive debugger that steps through the code argument (which
is an arbitrary expression or statement to be executed in the current
environment). The prompt is changed to indicate nested behavior. See
\ref{debugger:prompt}

\end{description}

\subsection{Starting/Restarting a Python Script ({\tt restart}, {\tt run})\label{subsection-restart}}

\begin{description}

\item[file \optional{\var{filename}}\label{command:file}]

Use \var{filename} as the Python program to be debugged.
It is compiled and becomes is the program executed when you use the `run'
command.  If no filename is given, this means to set things so there
is no Python file.

\item[restart \var{args...}]\label{command:restart}

Restart debugger and program via an \code{exec} call. All state
is lost, and new copy of the debugger is used.

Sometimes in debugging it is necessary to modify module code when one
finds a bugs in them. Python will not notice dynamically that a module
has changed and thus not reimport it (which also means that module
initialization code is not rerun either). So in such a situation one
must use \code{restart} rather than \code{run}.\footnote{It may be
possible to unimport by removing a the module from a namespace, but if
there are shared dynamically loaded objects those don't get unloaded.}

\item[run \var{args...}]\label{command:run}

Run or ``soft'' restart the debugged Python program. If a string is
supplied that becomes the new command arguments.  History,
breakpoints, actions and debugger options are preserved.  \code{R} is
a short command alias for \code{run}. 

You may notice that the sometimes you can \code{step} into modules
included via an \code{import} statement, but after a \code{run} this
stepping skips over the import rather than goes into it. A similar
situation is that you may have a breakpoint set inside class
\code{__init__} code, but after issuing \code{run} this doesn't seem
to get called---and in fact it isn't run again!

That's because in Python the \code{import} occurs only once. In fact,
if the module was imported \emph{before\/} invoking the program, you
might not be able to step inside an \code{import} the first time as
well.

In such a situation or other situations where \code{run} doesn't seem
to have the effect of getting module initialization code executed,
you might try using \code{restart} rather than \code{run}.

\end{description}

\subsection{Interfacing to the OS ({\tt cd}, {\tt pwd},
  {\tt shell})\label{subsection-os}}

\begin{description}

\item[cd \var{directory}]\label{command:cd}

Set working directory to \var{directory} for debugger and program
being debugged.

\item[pwd]\label{command:pwd}

Print working directory.  

\item[shell \var{statement}]\label{command:shell}

Execute the rest of the line as a shell command.


\end{description}

\subsection{Listing Program Code ({\tt list}, {\tt disassemble})\label{subsection-listing}}

\begin{description}

\item[disassemble \optional{\var{arg}}]

With no argument, disassemble at the current frame location.  With a
numeric argument, disassemble at the frame location at that line
number. With a class, method, function, code or string argument,
disassemble that.

\item[l(ist) \optional{- \Large{|} \var{first}\optional{, \var{last}}}]\label{command:list}

List source code.  Without arguments, list
\emph{n} lines centered around the current line or continue the previous
listing, where \emph{n} is the value set by \samp{set listsize} or
shown by \samp{show listsize}.  The default value is 10.

\samp{list -} lists \emph{n} lines before a previous listing. With one
argument other than `-', list \emph{n} lines centered around the
specified position.  With two arguments, list the given range; if the
second argument is less than the first, it is a count. \var{first} and
\var{last} can be either a function name, a line number, or
\var{filename}:\var{line-number}.

\end{description}

\subsection{Interfacing to the debugger ({\tt alias}, {\tt complete}, 
{\tt help}, {\tt quit}, {\tt kill}, {\tt source}, {\tt unalias})\label{subsection-misc}}

\begin{description}

\item[alias \optional{\var{name} \optional{command}}]\label{command:aliases}

Create an alias called \var{name} that executes \var{command}.  The
command must not be enclosed in quotes.  Replaceable parameters
can be indicated by \samp{\%1}, \samp{\%2}, and so on, while \samp{\%*} is
replaced by all the parameters.  If no command is given, the current
alias for \var{name} is shown. If no arguments are given, all
aliases are listed.

Aliases may be nested and can contain anything that can be legally
typed at the \code{pydb} prompt.  Note that internal \code{pydb}
commands can be overridden by aliases.  Such a command is then hidden
until the alias is removed.  Aliasing is recursively applied to the
first word of the command line; all other words in the line are left
alone.

As an example, here are two useful aliases (especially when placed
in the \code{.pydbrc} file):

\begin{verbatim}
#Print instance variables (usage "pi classInst")
alias pi for k in %1.__dict__.keys(): print "%1.",k,"=",%1.__dict__[k]
#Print instance variables in self
alias ps pi self
\end{verbatim}
                
\item[complete \var{command-prefix}]\label{command:complete}

If
\ulink{\module{readline}}{http://docs.python.org/lib/module-readline.html}
or one of readline-compatible interfaces such as
\ulink{\module{pyreadline}}{http://projects.scipy.org/ipython/ipython/wiki/PyReadline/Intro}
are available on your OS, the \code{complete} command will print a
list of command names that start with \var{command-prefix}.

\code{complete} will also work on \code{info}, \code{set}, and
\code{show} sub-command. 

In addition the command-completion key (usually the tab key) can be
used to complete command names, or \code{info}, \code{set}, and
\code{show} subcommands.

\item[h(elp) \optional{\var{command} \optional{subcommand}}]

Without argument, print the list of available commands.  With
\var{command} as argument, print help about that command.  \samp{help
pydb} displays the full documentation file; if the environment
variable \envvar{PAGER} is defined, the file is piped through that
command.  Since the \var{command} argument must be an identifier,
\samp{help exec} must be entered to get help on the \samp{!} command.

Some commands, \code{info}, \code{set}, and \code{show} can accept an
additional subcommand to give help just about that particular
subcommand. For example \code{help info line} give help about the
\code{info line} command.

\item[q(uit)]\label{command:quit}

Quit the debugger. The program being executed is aborted.

\item[kill \optional{unconditional}]\label{command:kill}

A non-maskable or ``hard'' kill. Basically, the program sends itself
\code{kill -9}. This may be needed to get out of the debugger when
doing thread debugging.

Since this is drastic, normally we prompt in interactive sessions
whether this is really what you want to do. If however
``unconditional'' is added, no questions are asked.

\item[source \var{filename}]\label{command:source}

Read commands from a file named \var{filename}.
Note that the file \code{.pydbrc} is read automatically
this way when \code{pydb} is started.

An error in any command terminates execution of the command and
control is returned to the console.

For tracking down problems with command files, see the \samp{set
cmdtrace on} debugger command. See \ref{command:cmdtrace}. 

\item[unalias \var{name}]\label{command:unalias}

Delete the specified alias.

\end{description}

\subsection{Signal handling ({\tt handle}, {\tt info handle}, {\tt signal})\label{subsection-signal}}

Partly as a result of the Matthew Fleming's Google 2006 Summer of Code
project, \code{pydb} contains signal handling similar to \code{gdb}.

A signal is an asynchronous event that can happen in a program.  Note:
only the main thread can intercept a signal, so if the main thread is
blocked, handling of the signal will be delayed.

The operating system defines the possible kinds of signals, and gives
each kind a name and a number.  For example, in Unix \code{SIGINT} is
the signal a program gets when you type an interrupt character (often
\code{C-c}); \code{SIGALRM} occurs when the alarm clock timer goes off
(which happens only if your program has requested an alarm).

When \code{pydb} changes signal handlers, it saves any value that the
debugged script might have installed. 

The debugger also installs an interrupt handler \code{SIGINT} so that
errant programs can be interrupted and you can find out where the
program was when you interrupted it.

Some signals, including \code{SIGALRM}, are a normal part of the
functioning of your program.  Others, such as \code{SIGSEGV}, indicate
errors; these signals are fatal--they kill your program immediately if the
program has not specified in advance some other way to handle the signal.
\code{SIGINT} does not indicate an error in your program, but it is normally
fatal so it can carry out the purpose of the interrupt: to kill the program.

\code{pydb} has the ability to detect any occurrence of a signal in
your program.  You tell \code{pydb} in advance what to do for each
kind of signal.  In the course of running the program, signal handlers
may be changed, and the debugger has the ability to watch for this
possibility and act accordingly. However since this adds a bit of
overhead it is turned off by default. \code{set sigcheck} and
\code{show sigcheck} can be used to set/show whether this checking is
done.

Normally, \code{pydb} is set up to let the non-erroneous signals like
\code{SIGALRM} be silently passed to your program (so as not to
interfere with their role in the program's functioning) but to stop
your program immediately whenever an error signal happens.  You can
change these settings with the \code{handle} command.

\subsubsection{Intercepting Signals ({\tt handle})}\label{subsubsection-handle}

\begin{description}
\item[handle \var{signal} \var{keywords}\ldots]
Change the way the debugger handles signal \var{signal}.  \var{signal}
can be the number of a signal or its name (with or without the
\code{SIG} at the beginning); The \var{keywords} say what change to
make.
\end{description}

The keywords allowed by the \code{handle} command can be abbreviated.
Their full names are:

\begin{description}
\item[stop]
The debugger should stop your program when this signal happens.  This implies
the \code{print} keyword as well.

\item[nostop]
The debugger should not stop your program when this signal happens.  It may
still print a message telling you that the signal has come in.

\item[print]
The debugger should print a message when this signal happens.

\item[noprint]
The debugger should not mention the occurrence of the signal at all.  

\item[pass]
The debugger should allow your program to see this signal; your program
can handle the signal, or else it may terminate if the signal is fatal
and not handled.  \code{pass} and \code{noignore} are synonyms.

\item[nopass]
The debugger should not allow your program to see this signal.
\code{nopass} and \code{ignore} are synonyms.

\item[stack]

The debugger should print a stack trace when it receives the
signal. When used with ``nostop'', this can be useful if you've got a
program that seems ``hung'' and you want to find out where it is, but
continue after showing this information.

\item[nostack]
The debugger should not print a stack trace when it receives the signal.
\end{description}

\subsubsection{Showing Signal-handling status ({\tt info handle}, {\tt
    info signals})}\label{command:info-handle}
\begin{description}
\item[info handle \optional{\var{signal-name}\code{\Large{|}}\var{signal-number}}]
If a signal name or number is given just the information regarding
that signal is shown. Here's an example:

\begin{verbatim}
  (Pydb) info handle INT
  Signal        Stop	Print	Print Stack	Pass to program
  SIGINT        True	True	False      	False
\end{verbatim}


If no signal name or number is given, print a table of all the kinds
of signals and how the debugger has been told to handle each one.  You
can use this to see the signal numbers of all the defined types of
signals.  

\code{info signal} is an alias for \code{info handle}.

\end{description}

\subsubsection{Sending your program a signal ({\tt signal})}
\begin{description}
\item[signal \{\var{signal-name} \code{\Large{|}} \var{signal-number}\}]
You can use the \code{signal} command send a signal to your
program. Supply either the signal name, e.g. \code{SIGINT}, or
\code{INT} or the signal number, e.g. \code{15}.
\end{description}

\subsection{Thread debugging ({\tt thread}, {\tt info thread}, {\tt
    qt}, {\tt frame}, {\tt break})\label{subsection-thread}}

Thread debugging is a bit experimental. By default it is not turned
on, since it adds a bit of additional complexity. 

To turn thread debugging on, pass the \code{--threading} option on the
\code{pydb} invocation.

Tests have shown that one can easily cause a deadlock in trying to
change the behavior if one is not careful. At present we don't have a
way to issue a \code{run} command, although the hard \code{restart}
works. See \ref{subsection-restart}. Also, to quit we haven't worked
out a sure-fire method to terminate gracefully like you can do when
there are no threads. We have however added a \code{kill} command. See
\ref{command:kill} for information on \code{kill}. This is a little
brutal, but at present it is about the best that can be done.

One will get the most information using Python 2.5 or having
\ulink{\module{threadframe}}{http://www.majid.info/mylos/stories/2004/06/10/threadframe.html}
installed with versions prior to 2.5. However since the debugger keeps
its own information about threads, one can still get some information
when this is not the case.

The debugger also relies on the
\ulink{\module{threading}}{http://docs.python.org/lib/module-threading.html}
module, especially to assist in tracing. If the program uses the
lower-level
\ulink{\module{thread}}{http://docs.python.org/lib/module-thread.html}
routines, some debugging can be done but it will be much more limited.

\subsubsection{Showing thread information ({\tt info thread})\label{command:info-thread}}

\begin{description}
\item[info thread \optional{\var{thread-name}}
  \optional{verbose\code{\Large{|}}terse}]

List all currently-known thread name(s).

If no thread name is given, we list info for all threads. Unless a
terse listing, for each thread we give:

\begin{itemize}
\item the class, thread name, and status as \code{<Class(Thread-n, status)>}
\item the top-most call-stack information for that thread. Generally
    the top-most calls into the debugger and dispatcher are omitted unless
    \code{set debug-pydb} is set to \code{True}.
\end{itemize}

The specific output you get depends on whether you are using Python
2.5 or later or (when not) have the \ulink{\module{threadframe}}{http://www.majid.info/mylos/stories/2004/06/10/threadframe.html} module installed. 

Here is an example assuming Python 2.5 is installed:

\begin{verbatim}
(Pydb) info thread 
----------------------------------------
   <_MainThread(MainThread, started)>
    <module>() called from file '/src/external-cvs/pydb/test/thread/q.py' at line 52
----------------------------------------
   <Producer(Thread-1, started)>
    run(self=<Producer(Thread-1, started)>) called from file '/src/external-cvs/pydb/test/thread/q.py' at line 19
----------------------------------------
-> <Consumer(Thread-2, started)>
    run(self=<Consumer(Thread-2, started)>) called from file '/src/external-cvs/pydb/test/thread/q.py' at line 39
----------------------------------------
   <Consumer(Thread-3, started)>
    run(self=<Consumer(Thread-3, started)>) called from file '/src/external-cvs/pydb/test/thread/q.py' at line 38
\end{verbatim}

The arrow \code{->} indicates which thread is current.  Here it is
Thread-2.  If \code{terse} is appended, we just list the thread name
and thread id. Here's an example:

\begin{verbatim}
(Pydb) info thread terse
   MainThread: -1210480976
   Thread-3: -1231336544
-> Thread-2: -1222943840
   Thread-1: -1214551136
\end{verbatim}

The arrow indicates the current thread; again it is Thread-2. This
listing is available on all Python versions and without threadframe
installed.

To get the full stack trace for a specific thread pass in the thread
name (assuming Python 2.5 or threadframe). Here's an example:

\begin{verbatim}
(Pydb) info thread Thread-3
    trace_dispatch_gdb(self=<threaddbg.threadDbg ins...) called from file 'None' at line 162
    run(self=<Consumer(Thread-3, star...) called from file 'q.py' at line 39
    __bootstrap(self=<Consumer(Thread-3, star...) called from file 'threading.py' at line 460
\end{verbatim}

If \code{verbose} appended without a thread name, then the entire
stack trace is given for each frame and the thread ID is shown as well.

\end{description}

%\subsubsection{Switching between threads ({\tt
%    thread})\label{command:thread}}
%
%You can switch between those threads that the debugger knows about and
%is stopped at a point inside the debugger.
%
%\begin{description}
%\item[thread \var{thread-name}] 
%Suspend the current thread and switch
%to the indicated thread. However the Python statement that was in
%progress before the debugger was entered will complete before the
%switch is done. 
%\end{description}

\subsubsection{Quitting a thread ({\tt qt})\label{command:qt}}
\begin{description}
\item[qt \var{thread-name}] 

Quit the specified thread. Use with caution; there may be other
threads be waiting on events that this thread can only give.
\end{description}

\subsubsection{Thread Extensions to the {\tt frame} and {\tt break} commands}\label{command:thread-extensions}

When thread debugging is turned on, a couple of normal \code{pydb}
commands allow for a thread name to be indicated. In particular, you
can specify whether or not a breakpoint should occur at a point across
all threads or only on a specific thread. 

For commands that take a thread name, you can use a dot (.) to
indicate the current frame.

And if Python version 2.5 or greater is installed or
\ulink{\module{threadframe}}{http://www.majid.info/mylos/stories/2004/06/10/threadframe.html}
with a older version of Python is installed, you can inspect and change
local variables from other thread frames via an expanded \code{frame}
command.

Note that the \code{next} and \code{step} commands do not guarantee
you will remain in the same thread. In particular \code{step} stops at
the next Python statement to be executed whichever thread that might
be.

There is currently a bug in \code{next} in that it is possible it may
act like \code{step} if occasionally.

\begin{description}
\item[frame \optional{\var{Thread-Name}} \var{frame-number}]
Move the current frame to the specified frame number. If a
\var{Thread-Name} is given, move the current frame to that. A dot (.) can be
  used to indicate the name of the current frame.

\item[b(reak)
  \optional{\optional{\var{filename}:}\var{lineno}\code{\Large{|}}\var{function}
  \optional{thread \var{thread-name}} \optional{, \var{condition}}}]
See \ref{command:break}. If a specific thread name is given then a
  breakpoint will occur only when in that thread. A dot (.) can be
  used to indicate the name of the current frame.

\item[tbreak
  \optional{\optional{\var{filename}:}\var{lineno}\code{\Large{|}}\var{function} \optional{thread \var{thread-name}}\optional{, \var{condition}}}]
\end{description}
See \ref{command:tbreak}. If a specific thread name is given then a
  breakpoint will occur only when in that thread. A dot (.) can be
  used to indicate the name of the current frame.

\section{The Debugger Module (\module{pydb}) and Class (\class{Pdb}) \label{pydb-module-class}}

The module \module{pydb} defines an interactive source code
debugger\index{debugging} for Python programs.

\file{pydb.py} can be invoked as a script to debug other scripts.  For
example:

\begin{verbatim}
python -m pydb myscript.py
\end{verbatim}

As with \code{pydb} invocation (see \ref{pydb-invocation}), \var{myscript.py}
must be fully qualified as a file, no path-searching is done to find
it. Also it must be Python source, not compiled or optimized
versions. Finally you may have to adjust \code{PYTHONPATH} if the
module \code{pydb} is not found.

When invoked as a script, \code{pydb} will automatically enter
post-mortem debugging if the program being debugged exits
abnormally. After post-mortem debugging (or after normal exit of the
program), \code{pydb} will restart the program. Automatic restarting
preserves \code{pydb}'s state (such as breakpoints) and in most cases
is more useful than quitting the debugger upon program's exit.

The debugger is extensible---it is defined as the class
\class{Pdb}\withsubitem{(class in pydb)}{\ttindex{Pdb}}.  When
creating a new \class{Pdb} object, command completion is available if
\ulink{\module{readline}}{http://docs.python.org/lib/module-readline.html}
or \module{pyreadline} is available. In such circumstance, one can
optionally specifying the key for command completion. The parameter
name here is \code{completekey} an its default value is the tab key.

The \class{Pdb} extension interface of \class{Gdb} a gdb-like debugger
command interface. This in turn inherts from \class{Cmd} and
\class{Bdb}; \class{Cmd} handles command-line aspects (e.g. keyboard
input, completion, help system, top-level command-invocation), while
\class{Bdb} handles debugger aspects (e.g. breakpoints, stepping, call
stack formatting). However \class{Cmd} and \class{Bdb} go through two
more modules \module{pycmd} and \module{pybdb} which are extensions of
\ulink{\module{cmd}}{http://docs.python.org/lib/module-cmd.html} and \module{bdb}
respectively. In general these extensions are used to paper over
deficiencies or differences in operation that are needed.

All of the methods intended for outside use are documented. In fact
the help documentation for commands comes from the document strings of
the corresponding methods.

One group of methods that may be useful to subclass would be the
output methods \method{errmsg}, \method{msg}, and
\method{msg_nocr}. In early regression development, I subclassed these
so that I could capture debugger output. It turned out, however, that
I needed far too many tests and working this way would not scale.

\subsection{Calling the Debugger from Inside your Program}\label{subsection-calling-pydb-inside-program}

When you issue \code{pydb myscript.py} you are running \code{pydb}
first which then invokes your script \code{myscript.py} via Python's
\code{exec} command. The debugger, \code{pydb}, tries hard to make
itself transparent and thus strips its own program options, resets
\code{sys.argv} and sets \code{__file__} to the values you would get
by calling the debugged program directly.

There may be however some subtle differences. For example the Python
interpreter used would be the one specified by \code{pydb} rather than
possibly that specified inside \code{myscript.py}. Also \code{pydb}
does not search your command path, as would be done if issued from a
shell.

Of course you can arrange to get the same interpeter by putting
\code{python} (with the right path) first before \code{pydb}; and you
can give an explicit file path in the script name to debug.

But there are times when even this won't work right.

There is another approach which obviates this complexity and the
attendant drawbacks. However in this approach though you need to
modify your program to add calls to the debugger at special
places. For this, the \code{pydb} function \code{set_trace()} (see
\ref{set-trace}) can be used. I've even this method to debug the
debugger itself and to debug regression tests.

When the \code{pydb.set_trace()} function is called, the program stops
before the next statement. To continue running the program, issue a
debugger \code{next}, \code{step} or a \code{continue} command.

The exit the program use the \code{quit} or a termanal EOF.

To make this more clear, let's go through an example. Save this in a
file called \code{hardtodebug.py}:

\begin{verbatim}
import pydb
# some code here
def test():
    # Force a call to the debugger in running code here
    pydb.set_trace()
    # ...
# ...
test()
x=5
\end{verbatim}

Now here's a sample run of the program:

\begin{verbatim}
python hardtodebug.py a b c
--Return--
--Return--
(/tmp/hardtodebug.py:9):
(Pydb) list
  4         # Force a call to the debugger in running code here
  5         pydb.set_trace()
  6         # ...
  7     # ...
  8     test()
  9  -> x=5
(/tmp/hardtodebug.py:9): 
(Pydb) restart
Re exec'ing
	['pydb', 'hardtodebug.py', 'a', 'b', 'c']
(/tmp/hardtodebug.py:1): 
(Pydb) 
\end{verbatim}

The first \code{--Return--} line printed is a result of the
\code{set_trace()} exiting. But note that we stopped \emph{after\/} the
return from \code{test()}, which explains the \emph{second\/}
\code{--Return--} line. Because the next executable statement is after
the implicit return. If you want to stop inside \code{test()} put a
statement after the \code{set_trace()}, such as a \code{return}
statement. Furthermore, if the program had ended at line 8, then
\emph{no\/} stopping would have occured becuase the end of the program
is reached first.\footnote{Perhaps this is a deficiency of the
debugger. I'm not sure what the right way to address though. One could
have another routine to stop at a return which would then skip
intervening statements. Another possibility is to add a routine to
have debugger stop inside the call, e.g. \code{set_trace} above. This
is not hard to code and I've done so, but I'm not sure that doesn't
just confuse things more.}

Also note that we can issue a \code{restart} which involves some
hackery. But at least in this case it gets things right.

There is one final advantage of using \code{set_trace()}.  When one is
stepping code or has put a breakpoint in code, the interpreter has to
be involved and calls debugger-checking code written in Python for
every statement that he debugged program runs. And this has a
noticeable effect. With \code{set_trace()} there is absolutely no
overhead (provided there are no other breakpoints set in the program).

\subsubsection{Line tracing}\label{subsubsection-linetracing}
Rather than trace the entire program, if there is a specific portion
that you want traced that can be done too, passing a list of debugger
commands to \code{set_trace}.

\begin{verbatim}
import pydb 
# some code here ...
pydb.set_trace(["set linetrace on", "continue"])
# This and subsequent lines will be traced
# more code ...
pydb.set_trace(["set linetrace off", "continue"])
\end{verbatim}

\subsection{Calling the Debugger after a Crash (Post-Mortem Debugging)}\label{subsection-post-mortem}

It is also to possible enter the debugger after a crash or traceback
even though the program was not started via \code{pydb}. This is
called \emph{post-mortem debugging}.

There are in fact three possibilities. The first way does not require
modification to your program but does require that the program be run
from the Python interpreter shell. The second possibility doesn't
require one to be inside the Python interpreter shell, but does require
modification to your program. The third possibility is if you had
started using \code{pydb}.

\subsubsection{Post-Mortem Debugging within a Python Interpreter Shell
({\tt pydb.pm()})}
Here, all you do is \code{import pydb} (if that hasn't
been done already), and call \code{pydb.pm()}.

To make this more concrete we will give an example. We have the
following text \code{mymodule.py}

\begin{verbatim}
def test():
    print spam
\end{verbatim}

Now here's a sample session

\begin{verbatim}
>>> import mymodule
>>> mymodule.test()
Traceback (most recent call last):
  File "<stdin>", line 1, in ?
  File "mymodule.py", line 2, in test
    print spam
NameError: global name 'spam' is not defined
>>> import pydb
>>> pydb.pm()
(/home/src/external-cvs/pydb/test/mymodule.py:2):  test
(Pydb) where
-> 0 test() called from file '/tmp/mymodule.py' at line 2
## 1 in file '<stdin>' at line 1
(Pydb) list
  1     def test():
  2  ->     print spam
[EOF]
(Pydb) quit
>>> 
\end{verbatim}

At present if you are using \code{ipython}, that captures the
exception and \code{sys.last_traceback} will not be defined.

If you have a traceback stored say in variable \code{t}, instead of 
\code{pydb.pm()} above, use \code{pydb.post_mortem(t)}.

\subsubsection{Post-Mortem Debugging arranged within a Python Script
  ({\tt pydb.exception_hook})}

If you didn't start your program inside a Python interpreter shell,
there you can still get into the debugger when an exception
occurs. Here you need to arrange for this possibility adding to the
script these lines:

\begin{verbatim}
  import sys
  sys.excepthook = pydb.exception_hook
\end{verbatim}


\subsubsection{Post-Mortem Debugging within {\tt pydb}}
When invoked as a script, \code{pydb} will automatically enter
post-mortem debugging if the program being debugged exits
abnormally. 

After post-mortem debugging (or after normal exit of the
program), \code{pydb} will restart the program.  Automatic restarting
preserves \code{pydb}'s state (such as breakpoints) and in most cases
is more useful than quitting the debugger upon the program's exit.

\subsection{Entering the Debugger from Outside the Program}
It's possible to set things up so that when you send a signal to your
program you enter the debugger. Up until that point though there will
be no overhead associated with the debugger.

To do this, you need to create an instance of the
\class{SignalManager} object and call that object's \code{action}
method passing a string exactly like you would for the 
debugger's \code{handle} command. Here's an example:

\begin{verbatim}
  ...
  from pydb.sighandler import SignalManager
  ...
  h = SignalManager()
  h.action('SIGUSR1 stack print stop')
  ...

\end{verbatim}

In the above example, a signal handler is set upt so that when the
\code{SIGUSR1} signal is sent, we will print a message, show the stack
trace and then stop inside the debugger. If you don't want to stop
inside the debuber bug just see where you are omit the ``stop''
keyword or change it it ``nostop''.

See \ref{subsubsection-handle} for information parameters names used
in the string and their meanings.

\subsection{Entering the Debugger from Python or a Python
  Shell}\label{subsection-entering-pydb-from-python}

If you want to debug a Python script from inside another Python
program (often a Python shell program such as
\ulink{IPython}{http://ipython.scipy.org/moin/ipython}), the 
\code{runl()}, and \code{runv()} function calls can be of help. See
\ref{entry-fns} for information about these calls. 

The 'l', and 'v' suffixes have meanings analogous that used in
\code{os.spawnv()}, \code{os.execv()}, \code{os.spawnl()}, or
\code{os.execl()} (and other Python library functions).

\subsection{Yet Another Method of Invocation}

With the caveat mentioned in
\ref{subsection-calling-pydb-inside-program} about problems with using
\code{pydb} initially to run a script, I think most people will
probably use the \code{pydb} command described in
\ref{pydb-invocation}. However for completeness here we give another
alternative.

Here's another way run a program under control of the debugger:

\begin{verbatim}
>>> import pydb
>>> import mymodule
>>> pydb.run('mymodule.test()')
(<string>:1): 
(/usr/lib/python2.4/bdb.py:366):  run
(Pydb) continue
\end{verbatim}

One difference betwen this kind of invocation that of Section
\ref{pydb-invocation} is that statements may get executed in the
import of \code{mymodule} will not be debugged. For example if
\code{mymodule} implements a standalone program and/or has a
\code{__name__ = '__main__' } clause, you probably don't want to use
this method.


\subsection{Inheritance from class \class{Cmd}}
Because \class{Pdb} inherts from \class{Cmd}, the following
conventions are used. 

All of the debugger commands listed in \ref{command:debugger} are methods
of a \class{Pdb} object. The method names are the command name prefixed by
\code{do_}. For example the method handling the \samp{step} command is
\code{do_step}, and the command handling the \samp{frame} command is
\code{do_frame}. 

If you have a \class{Pdb} object, it is possible to call any of the
commands listed in \ref{pydb-commands} directly. Parameters needed by
the methods (for example breakpoint numbers for the \code{enable}
command), are passed as a single string argument. The string contains
the values of the parameters and multiple parameters are separated by
spaces. Each method parses out parameters and performs needed
conversions.  For example, a \class{Pdb} object, \code{p}, can enable
breakpoint numbers 3, 5, and 10 like this: \code{p.do_enable("3 5
10")}, and this has the same effect as if \code{"enable 3 5 10"} were
issued as a debugger command. String parameters should not have
additional quotes in the string. For example to set the filename where
history commands are to be saved (\ref{command:hist-filename}), the
method call would be \code{p.do_set("history filename
/tmp/myhistfile")} without any quotes around \code{/tmp/myhistfile}.

Also inherited \class{Cmd}, is the help mechanism, although some
customization has been made to allow for subcommand help.  See the
\ulink{Python cmd module}{http://docs.python.org/lib/module-cmd.html}
for more information.

If readline support is available on your OS, \class{Cmd} will use that
and both command history and command completion will be available. In
addition, the debugger \code{complete} command is defined (see
\ref{command:complete}). The \code{complete} command for example is
used internally by GNU Emacs debugger \code{gud}.

\subsection{Debugger Entry Functions}\label{entry-fns}

The \code{pydb} module defines the following functions, and each enters
the debugger in a slightly different way:

\begin{funcdesc}{exception_hook}{type, value, tb, dbg_cmds=None,
    cmdfile=None}

An exception hook to call \code{pydb}'s post-mortem debugger.

\code{cmdfile} is an optional debugger command file you want to 
\code{source} commands on.

\code{dbg_cmds} is a list of debugger commands you want to run.

To use add this to your Python program:
\begin{verbatim}
       import sys
       sys.excepthook = pydb.exception_hook
\end{verbatim}
\end{funcdesc}

\begin{funcdesc}{pm}{\optional{dbg_cmds=None}}
Enter post-mortem debugging of the traceback found in
\code{sys.last_traceback}. Note you may need to have \code{sys}
imported priort to having the error raised to have sys.last_traceback
set.

You can run debugger commands by passing this as a list as parameter
\code{dbg_cmds}. For example if you want the display listsize to be 20
by default on entry pass \code{["set listsize 20",]}.
\end{funcdesc}

\begin{funcdesc}{post_mortem}{traceback\optional{dbg_cmds=None,
  cmdfile=None}}\label{post-mortem}
Enter post-mortem debugging of the given \var{traceback} object.

You can run debugger commands by passing this as a list as parameter
\code{dbg_cmds}. For example if you want the display listsize to be 20
by default on entry pass \code{["set listsize 20",]}.

\end{funcdesc}

\begin{funcdesc}{run}{statement\optional{, globals\optional{, locals}}}
Execute the \var{statement} (given as a string) under debugger
control starting with the statement subsequent to the place that the
this appears in your program.

The debugger prompt appears before any code is executed; you can set
breakpoints and type \samp{continue}, or you can step through the
statement using \samp{step} or \samp{next} See \ref{command:step} and
\ref{command:next} for explanations of these commands.  The optional
\var{globals} and \var{locals} arguments specify the environment in
which the code is executed; by default the dictionary of the module
\refmodule[main]{__main__} is used.  (See the explanation of the
\keyword{exec} statement or the \function{eval()} built-in function.)

Note that this is not at all like pydb's (or GDB's) \code{run} (see
\ref{command:run}) debugger command. This function should be called
from inside the program you are trying to debug.
\end{funcdesc}

\begin{funcdesc}{runcall}{function\optional{, argument, ...}}
Call the \var{function} (a function or method object, not a string)
with the given arguments starting with the statement subsequent to the
place that the this appears in your program..

When \function{runcall()} returns, it returns
whatever the function call returned.  The debugger prompt appears as
soon as the function is entered.
\end{funcdesc}

\begin{funcdesc}{runeval}{expression\optional{, globals\optional{, locals}}}
Evaluate the \var{expression} (given as a string) under debugger
control starting with the statement subsequent to the place that the
this appears in your program.

When \function{runeval()} returns, it returns the value of the
expression.  Otherwise this function is similar to
\function{run()}.
\end{funcdesc}

\begin{funcdesc}{runl}{\var{arg1} \optional{, \var{arg2}\ldots}}
This is a way to call the debugger on a Python script (usually not the
same one as is currently running) from inside Python. In particular
this is useful inside Python shells.  The parameters form the options
to the debugger as well as the Python script to debug and options to
pass to that.

For example:
\begin{verbatim}
        import pydb
        pydb.runl("--threading", "--nx",
                  "myscript", "--my-first-option" "myfirstarg")
\end{verbatim}

Here \code{--threading} and \code{--nx} go to the debugger and
\code{myscript} is the name of the Python program to run which gets the
remaining parameters, \code{--my-first-option} and \code{myfirstarg}.

The 'l', suffix to indicate a list parameter is analogous to the use
in \code{os.spawnl()}, or \code{os.execl()}.

If you want to debug the current Python script, use \code{set_trace()}
described below.

\end{funcdesc}

\begin{funcdesc}{runv}{\var{args}}
This is a way to call the debugger on a Python script (usually not the
same one as is currently running) from inside Python. In particular
this is useful inside Python shells.  The single parameter is a list
of the options to the debugger as well as the Python script to debug
and options to pass to that.

\code{runl(*args)} is also the same as \code{runv((args))}.

For example:
\begin{verbatim}
        import pydb
        args=("--threading", "--nx",
              "myscript", "--my-first-option" "myfirstarg")
        pydb.runv(args)
\end{verbatim}

Here \code{--threading} and \code{--nx} go to the debugger and
\code{myscript} is the name of the Python program to run which gets the
remaining parameters, \code{--my-first-option} and \code{myfirstarg}.

The 'v', suffix to indicate a variable number of parameters is analogous to
the use in \code{os.spawnv()}, or \code{os.execv()}.

If you want to debug the current Python script, use \code{set_trace()}
described below.
\end{funcdesc}

\begin{funcdesc}{set_trace}{\optional{dbg_cmds=None,
      add_exception_hook=True, add_threaddbg=False}}\label{set-trace}
Enter the debugger at the statement which follows (in execution) the
\code{set_trace()} statement. This hard-codes a call to the debugger
at a given point in a program, even if the code is not otherwise being
debugged. For example you might want to do this when an assertion
fails.

It is useful in a couple of other situations. First, there may be some
problem in getting the debugger to stop at this particular place for
whatever reason (like flakiness in the debugger). Alternatively, using
the debugger and setting a breakpoint can slow down a program a
bit. But if you use this instead, the code will run as though the
debugger is not present.

When the debugger is quitting, this causes the program to be
terminated. If you want the program to continue instead, use the
\code{debugger} function.

You can run debugger commands by passing this as a list in parameter
\code{dbg_cmds}.  Unless \code{add_exception_hook} is set to
\code{False}, we install an exception hook to enter the debugger on
any otherwise unhandled exception. 

If you want experimental thread debugging support, set
\code{add_threaddbg} to \code{True}.

\end{funcdesc}

\section{How the Debugger Works \label{debugger-hooks}}

Some changes were made to the interpreter:

\begin{itemize}
\item \code{sys.settrace(\var{func})} sets the global trace function
\item there can also be a local trace function (see below)
\end{itemize}

Trace functions have three arguments: \var{frame}, \var{event}, and
\var{arg}. \var{frame} is the current stack frame.  \var{event} is a
string: \code{'call'}, \code{'line'}, \code{'return'}, \code{'exception'},
 \code{'c_call'}, \code{'c_return'}, or \code{'c_exception'}. \var{arg}
 depends on the event type.

The global trace function is invoked (with \var{event} set to
\code{'call'}) whenever a new local scope is entered; it returns
a reference to the local trace function to be used in that scope, or
\code{None} if the scope couldn't be traced.

The local trace function  returns a reference to itself (or to
another function for further tracing in that scope), or \code{None} to
turn off tracing in that scope.

Instance methods are accepted (and very useful) as trace functions.

The events have the following meanings:

\begin{description}

\item[\code{'call'}]
A function is called (or some other code block entered).  The global
trace function is called; \var{arg} is \code{None};
the return value specifies the local trace function.

\item[\code{'line'}]
The interpreter is about to execute a new line of code (sometimes
multiple line events on one line).  The local trace function is
called; \var{arg} is \code{None};\ the return value specifies the new
local trace function.

\item[\code{'return'}]
A function (or other code block) is about to return.  The local trace
function is called; \var{arg} is the value that will be returned.  The
trace function's return value is ignored.

\item[\code{'exception'}]
An exception has occurred.  The local trace function is called;
\var{arg} is a triple \code{(\var{exception}, \var{value},
\var{traceback})};\ the return value specifies the new local trace
function.

\item[\code{'c_call'}]
A C function is about to be called.  This may be an extension function
or a builtin.  \var{arg} is the C function object.

\item[\code{'c_return'}]
A C function has returned. \var{arg} is \code{None}.

\item[\code{'c_exception'}]
A C function has thrown an exception.  \var{arg} is \code{None}.

\end{description}

Since an exception is propagated down the chain of callers, an
\code{'exception'} event is generated at each level.

For more information on code and frame objects, see to the
\ulink{Python Reference
Manual}{http://docs.python.org/ref/ref.html}.

\section{Files making up the Debugger \label{debugger-files}}

Here are the files making up the debugger and what is in
them. Currently we use GNU automake and makes substitutings in files
and Python programs which then get installed. A file with a \code{.in}
suffix is a file that is the input for the substitution. The
corresponding output file has the \code{.in} stipped off. For example
{\tt pydb.py.in} becomes \code{pydb.py} after a Python interpreter and
package name (pydb) is substituted inside it.

\begin{description}
  
\item[pydb.py.in]
Python debugger. Contains user-callable routines, e.g. 
\code{run}, \code{set_trace}, \code{runeval}.

\item[display.py]

Classes to support gdb-like display/undisplay for pydb, the Extended
Python debugger. Class Display and DisplayNode are defined.

\item[fns.py]

Functions to support the Debugger.

\item[gdb.py.in]

Handles gdb-like command processing.

\item[info.py]

Info subcommands. (Well, most of them).

\item[pydbbdb.py]

Routines here have to do with the subclassing of bdb.  Defines Python
debugger Basic Debugger (Bdb) class.  This file could/should probably
get merged into bdb.py

\item[pydbcmd.py]

Routines here have to do with parsing or processing commands,
generally (but not always) they are not specific to pydb. They are sort
of more oriented towards any gdb-like debugger. Also routines that need to
be changed from cmd are here.

\item[set.py]

Set subcommands. (Well, most of them).

\item[show.py]

Show subcommands. (Well, most of them).

\item[sighandler.py]

Classes and routines which help in signal handling. Of the classes
there is a signal handler manager class, \code{SignalManager}, and a
class that is used by the signal handler that the program sees,
\code{SigHandler}.

\item[subcmd.py]

Implements a sub-command class for handling \code{info}, \code{set},
and \code{show} subcommand processing.

\item[threaddbg.py]

Implements thread debugging.

\end{description}

\section{Installation}

The Python code is available from the \ulink{pydb files section on
sourceforge.net}{http://sourceforge.net/project/showfiles.php?group_id=61395&package_id=175827}.
{\em If you have made an OS package and would like you're URL listed
here, let me know}.

\subsection{Installation options}

The program is not configured using \code{setup.py} yet, but
\code{configure}. Some configuration options that may be of interest.

\begin{description} 

\item[--{}--with--{}--python]

Normally the \code{configure} uses your execution search-path variable
(\code{PATH}) to find the python interpreter. However it is possible
that you might now have python or the right python in your
\code{PATH}.  Soome installations may have several versions of python
may be installed or installed in an unusual place. Use this
configuration in such cases.

\item[--{}--with--{}--lispdir]

This option overrides where to put files which can be used by GNU Emacs. It
should be a place that GNU Emacs users will have listed in the
\code{load-path} inside GNU Emacs.

\item[--{}--with--{}--site-packages]

This option overrides where to put files the \code{pydb} package.  It
should be a place that python searches when modules are
\code{import}'ed.

\item[--{}--enable--{}--pyreadline]

If you don't have readline but have \code{pyreadline} or prefer to use
that, use this option.

\end{description}

